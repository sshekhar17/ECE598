\documentclass[12pt]{article}
\usepackage[english]{babel}
\usepackage[utf8x]{inputenc}
\usepackage[T1]{fontenc}
\usepackage{../scribe}
\usepackage{listings}

\usepackage{hyperref}
% \usepackage{cleveref}
\usepackage{xcolor}
\usepackage{natbib}
\usepackage{tikz}
\renewcommand{\arraystretch}{1.5}
\usepackage{nicefrac}
% \newcommand{\thetahat}{\widehat{\theta}}
% \Scribe{Shubhanshu Shekhar}
\Lecturer{Shubhanshu Shekhar}
\LectureNumber{7}
\LectureDate{18th September, 2025}
\LectureTitle{Minimax Lower Bounds Part II}

\lstset{style=mystyle}
% \usepackage{tikz}
\usepackage{comment}
% \usepackage{pgfplots}
% \usepackage{enumitem}
\usepackage{amsmath,amssymb,amsthm,amsfonts,latexsym,bbm,xspace,graphicx,float,mathtools,mathdots}
\usepackage{braket,caption,ellipsis,xcolor,textcomp}
\usepackage{combelow} %for you, Rusins Freivalds
% \usepackage[backref,colorlinks,citecolor=blue,bookmarks=true]{hyperref}
\usepackage{booktabs}
\usepackage{hyperref}
% \usepackage[capitalize,noabbrev]{cleveref}
% \crefname{ineq}{inequality}{inequalities}
% \creflabelformat{ineq}{#2{\upshape(#1)}#3}

% removed the next two oas it is preloaded in neurips_2022.sty
\usepackage[letterpaper,margin=1in]{geometry}
\usepackage{enumitem} 
%\usepackage[protrusion=true,expansion=true]{microtype}   % slows compilation a loeve


% environments
% \newtheorem{theorem}{Theorem}[section]

\newtheorem{question}[theorem]{Question}
\newtheorem{answer}[theorem]{Answer}
\newtheorem{hypothesis}[theorem]{Hypothesis}
\newtheorem{assumption}[theorem]{Assumption}
\newtheorem{result}[theorem]{Main Result}
\newtheorem{problem}[theorem]{Problem}



\theoremstyle{definition}
\newtheorem{examples}[theorem]{Example}



\newcommand{\dist}{\textnormal{dist}}
\newcommand{\vol}{\textnormal{vol}}
\newcommand{\spn}{\textnormal{span}}
\newcommand{\supp}{\textnormal{supp}}
\newcommand{\tr}{\operatorname{tr}}
\newcommand{\codim}{\operatorname{codim}}
\newcommand{\diag}{\operatorname{diag}}

\newcommand{\bbone}{\mathbbm 1}
\newcommand{\Id}{\bbone}

% complexity classes
\newcommand{\PTIME}{\mathsf{P}}
\newcommand{\LOGSPACE}{\mathsf{L}}
\newcommand{\ZPP}{\mathsf{ZPP}}
\newcommand{\RP}{\mathsf{RP}}
\newcommand{\TC}{\mathsf{TC}}
\newcommand{\AC}{\mathsf{AC}}
\newcommand{\SC}{\mathsf{SC}}
\newcommand{\SZK}{\mathsf{SZK}}
\newcommand{\AM}{\mathsf{AM}}
\newcommand{\IP}{\mathsf{IP}}
\newcommand{\PSPACE}{\mathsf{PSPACE}}
\newcommand{\EXP}{\mathsf{EXP}}
\newcommand{\MIP}{\mathsf{MIP}}
\newcommand{\NEXP}{\mathsf{NEXP}}
\newcommand{\BQP}{\mathsf{BQP}}

% short forms
\newcommand{\dleta}{\delta}
\newcommand{\simga}{\sigma}
\newcommand{\vphi}{\varphi}
\newcommand{\ol}[1]{\overline{#1}}
\newcommand{\ul}[1]{\underline{#1}}
\newcommand{\ot}{\otimes}
\newcommand{\zo}{\{0,1\}}
\newcommand{\co}{:}                       %\newcommand{\co}{\colon}
\newcommand{\bdry}{\partial}
\newcommand{\grad}{\nabla}
\newcommand{\symmdiff}{\triangle} \newcommand{\symdiff}{\symmdiff}


% calligraphic letters
% bold
\newcommand{\bone}{\boldsymbol{1}}
\newcommand{\bbeta}{\boldsymbol{\beta}}
\newcommand{\bdelta}{\boldsymbol{\delta}}
\newcommand{\bepsilon}{\boldsymbol{\epsilon}}
\newcommand{\blambda}{\boldsymbol{\lambda}}
\newcommand{\bomega}{\boldsymbol{\omega}}
\newcommand{\bpi}{\boldsymbol{\pi}}
\newcommand{\bphi}{\boldsymbol{\phi}}
\newcommand{\bvphi}{\boldsymbol{\varphi}}
\newcommand{\bpsi}{\boldsymbol{\psi}}
\newcommand{\bsigma}{\boldsymbol{\sigma}}
\newcommand{\btheta}{\boldsymbol{\theta}}
\newcommand{\btau}{\boldsymbol{\tau}}
\newcommand{\ba}{\boldsymbol{a}}
\newcommand{\bb}{\boldsymbol{b}}
\newcommand{\bc}{\boldsymbol{c}}
\newcommand{\bd}{\boldsymbol{d}}
\newcommand{\be}{\boldsymbol{e}}
\newcommand{\boldf}{\boldsymbol{f}}
\newcommand{\bg}{\boldsymbol{g}}
\newcommand{\bh}{\boldsymbol{h}}
\newcommand{\bi}{\boldsymbol{i}}
\newcommand{\bj}{\boldsymbol{j}}
\newcommand{\bk}{\boldsymbol{k}}
\newcommand{\bell}{\boldsymbol{\ell}}
\newcommand{\bp}{\boldsymbol{p}}
\newcommand{\bq}{\boldsymbol{q}}
\newcommand{\br}{\boldsymbol{r}}
\newcommand{\bs}{\boldsymbol{s}}
\newcommand{\bt}{\boldsymbol{t}}
\newcommand{\bu}{\boldsymbol{u}}
\newcommand{\bv}{\boldsymbol{v}}
\newcommand{\bw}{\boldsymbol{w}}
\newcommand{\bx}{{\boldsymbol{x}}}
\newcommand{\by}{\boldsymbol{y}}
\newcommand{\bz}{\boldsymbol{z}}
\newcommand{\bA}{\boldsymbol{A}}
\newcommand{\bB}{\boldsymbol{B}}
\newcommand{\bC}{\boldsymbol{C}}
\newcommand{\bD}{\boldsymbol{D}}
\newcommand{\bE}{\boldsymbol{E}}
\newcommand{\bF}{\boldsymbol{F}}
\newcommand{\bG}{\boldsymbol{G}}
\newcommand{\bH}{\boldsymbol{H}}
\newcommand{\bI}{\boldsymbol{I}}
\newcommand{\bJ}{\boldsymbol{J}}
\newcommand{\bL}{\boldsymbol{L}}
\newcommand{\bM}{\boldsymbol{M}}
\newcommand{\bP}{\boldsymbol{P}}
\newcommand{\bR}{\boldsymbol{R}}
\newcommand{\bS}{\boldsymbol{S}}
\newcommand{\bT}{\boldsymbol{T}}
\newcommand{\bU}{\boldsymbol{U}}
\newcommand{\bV}{\boldsymbol{V}}
\newcommand{\bW}{\boldsymbol{W}}
\newcommand{\bX}{\boldsymbol{X}}
\newcommand{\bY}{\boldsymbol{Y}}
\newcommand{\bZ}{\boldsymbol{Z}}

% bold calligraphic
\newcommand{\bcalG}{\boldsymbol{\calG}} \newcommand{\calbG}{\bcalG}
\newcommand{\bcalX}{\boldsymbol{\calX}} \newcommand{\calbX}{\bcalX}
\newcommand{\bcalY}{\boldsymbol{\calY}} \newcommand{\calbY}{\bcalY}
\newcommand{\bcalZ}{\boldsymbol{\calZ}} \newcommand{\calbZ}{\bcalZ}

% left-right wrappers
\DeclarePairedDelimiter\parens{\lparen}{\rparen}
% words -- there's supposed to be something better than xspace though
\newcommand{\Hastad}{H{\aa}stad\xspace}
\newcommand{\Holder}{H{\"o}lder\xspace}
\newcommand{\ErdosRenyi}{Erd\H{o}s--R\'{e}nyi\xspace}

%%%%%%%%%%%%% hyphenation %%%%%%%%%%%%%
\hyphenation{Naka-shima}


%%%%%%%%%%%% document-writing macros %%%%%%%%%%%%

%\newcommand{\ignore}[1]{*\marginpar{\tiny \em ignored part}}



% \newcommand{\defined}{\coloneqq}
\usepackage{autonum}


% 
\usepackage{natbib}
\setcitestyle{authoryear,open={(},close={)}} %Citation-related commands






% % to avoid loading the natbib package, add option nonatbib:
% % \usepackage[nonatbib]{neurips_2020}
% \usepackage[utf8]{inputenc} % allow utf-8 input
% \usepackage[T1]{fontenc}    % use 8-bit T1 fonts
% \usepackage{hyperref}       % hyperlinks
% \usepackage{url}            % simple URL typesetting
% \usepackage{booktabs}       % professional-quality tables
% \usepackage{amsfonts}       % blackboard math symbols
% \usepackage{nicefrac}       % compact symbols for 1/2, etc.
% \usepackage{microtype}      % microtypography
% \usepackage{mathtools}




% % --------------------------------
% % Additional packages and commands
% \usepackage{amsmath,amsfonts,amsthm,amssymb}
% \usepackage[]{xcolor}
% \usepackage[]{natbib}

% \usepackage{cleveref}       % hyperlinks

% \newcommand{\RR}{\mathbb{R}}
\newcommand{\NN}{\mathbb{N}}
\newcommand{\EE}{\mathbb{E}}
% \newcommand{\Xcal}{\mathcal{X}}
% \newcommand{\Ycal}{\mathcal{Y}}
% \newcommand{\Dcal}{\mathcal{D}}
% \newcommand{\Lcal}{\mathcal{L}}
% \newcommand{\Ncal}{\mathcal{N}}
% \newcommand{\Fcal}{\mathcal{F}}
% \newcommand{\GP}{\textnormal{GP}}
% \newcommand{\williex}[1]{\textcolor{red}{[#1]}}
\DeclareMathOperator*{\argmin}{argmin}
\DeclareMathOperator*{\argmax}{argmax}
\DeclareMathOperator*{\esssup}{esssup}

% \newtheorem{lemma}{Lemma}
% \newtheorem{theorem}{Theorem}
% \newtheorem{proposition}{Proposition}
% \newtheorem{corollary}{Corollary}

% \theoremstyle{definition}
% \newtheorem{definition}{Definition}
% \newtheorem{remark}{Remark}
% \newtheorem{example}{Example}
% \newtheorem{assumption}{Assumption}



% \usepackage{graphicx} 
% \usepackage{caption}
% \usepackage{subcaption}
% \usepackage{enumitem}
% % --------------------------------
% \usepackage{xspace}

\newcommand{\iid}{\text{i.i.d.}\xspace}
\newcommand{\prob}{\mathbb{P}}
\newcommand{\expec}{\mathbb{E}}
\newcommand{\simiid}{\stackrel{i.i.d.}{\sim}}


%=======================================
\newcommand{\mc}[1]{\mathcal{#1}}
\newcommand{\mbb}[1]{\mathbb{#1}}
\newcommand{\lb}{\left[}
\newcommand{\rb}{\right]}
\newcommand{\lp}{\left(}
\newcommand{\rp}{\right)}
%=======================================


%=======================================
\newcommand{\defined}{\coloneqq}


%=======================================
\newcommand{\arl}{\text{ARL}\xspace} %average run length 
\newcommand{\delay}{D\xspace} % detection delay
\newcommand{\hatT}{\widehat{T}} %estimated change point 
\newcommand{\hatepsilon}{\widehat{\epsilon}} % estimated change magnitude 
\newcommand{\barB}{\bar{B}} % time-reversed confidence sequence
\newcommand{\cusum}{\text{CuSum}\xspace}

\newcommand{\backwardtest}{\tau^{\text{back}}}
\newcommand{\dkl}{D_{\text{KL}}}
\newcommand{\dks}{d_{\text{KS}}}
\newcommand{\dmmd}{d_{\text{MMD}}}
\newcommand{\red}[1]{\textcolor{red}{#1}}
\newcommand{\blue}[1]{\textcolor{blue}{#1}}
\newcommand{\thetahat}{\widehat{\theta}}
\newcommand{\Xhat}{\widehat{X}}
\newcommand{\muhat}{\widehat{\mu}}
\newcommand{\sigmahat}{\widehat{\sigma}}
\newcommand{\sigmatilde}{\widetilde{\sigma}}
\newcommand{\Phat}{\widehat{P}}
\newcommand{\Qhat}{\widehat{Q}}
\newcommand{\fcsdetector}{\texttt{FCS-Detector}\xspace}
\newcommand{\bcsdetector}{\texttt{BCS-Detector}\xspace}
\newcommand{\rfcsdetector}{\texttt{RCS-Detector}\xspace}
%%% filtration 
\newcommand{\filtration}{\mathcal{F}}

\newcommand{\uniform}{\mathrm{Uniform}}
\newcommand{\mse}{\mathrm{MSE}}
\newcommand{\power}{\mathrm{Power}}
\newcommand{\Ytilde}{\widetilde{Y}}
\newcommand{\mhat}{\widehat{m}}
\newcommand{\tauhat}{\widehat{\tau}}
\newcommand{\thetatilde}{\widetilde{\theta}}
\newcommand{\qtilde}{\widetilde{q}}
\newcommand{\Xtilde}{\widetilde{X}}
\newcommand{\nptest}{\Psi^*_{\mathrm{NP}}}
\newcommand{\umptest}{\psi^*_{\mathrm{UMP}}}
\newcommand{\RTY}{R_{\Theta, Y}}
\newcommand{\PTY}{P_{\Theta, Y}}
\newcommand{\RR}{\boldsymbol{R}}
\newcommand{\Hab}{\mathcal{H}_a^b} 
\newcommand{\Yab}{Y_a^b}
\newcommand{\boldw}{\boldsymbol{w}}
\newcommand{\lpost}{L_{\text{post}}}
\newcommand{\cramer}{Cram\'er}
\newcommand{\Beta}{\mathrm{Beta}}
\newcommand{\Bernoulli}{\mathrm{Bernoulli}}
\newcommand{\Binomial}{\mathrm{Binomial}}
\newcommand{\GammaDist}{\mathrm{Gamma}}
\newcommand{\data}{X=(Y_1, \ldots, Y_n)}
\newcommand{\convdist}{\stackrel{d}{\rightarrow}}
\newcommand{\convprob}{\stackrel{p}{\rightarrow}}
\newcommand{\typicalset}[1]{{#1}_\epsilon^{(n)}}
\newcommand{\vhat}{\hat{\boldsymbol{v}}}
\newcommand{\boldX}{\boldsymbol{X}}

\newcommand{\boldtheta}{\boldsymbol{\theta}}


\begin{document}
	\MakeScribeTop
%#############################################################
%#############################################################
%#############################################################
%#############################################################

In this lecture, we will study a generalization of the two-point method in which the second point is replaced with a mixture. This minor change makes this method significantly more potent in testing problems involving a ``$k$-subset'' structure and for functional estimation tasks. 

\section{Generalized Two-Point Method}
\label{sec:generalized-two-point-method}
As before, we are working in a decision-theoretic setting with model $\{P_\theta: \theta \in \Theta\}$, decision space $\calW$, and a loss function $L:\Theta \times \calW \to \mathbb{R}$. 

\begin{theorem}
	\label{theorem:point-vs-mixture} Suppose there exist $\theta_0 \in \Theta$, and $\Theta_1 \subset \Theta$, satisfying the following \emph{uniform separation} condition with some $\omega>0$: 
	\begin{align}
	\inf_{w \in \calW} \inf_{\theta_1 \in \Theta_1} \frac{L(\theta_0, w) + L(\theta_1, w)}{2} \geq \omega. 
	\end{align}
	Let $\mu$ denote any probability measure supported on $\Theta_1$, and let $P_\mu$ denote the mixture distribution satisfying 
	\begin{align}
	P_\mu(E) = \int_{\Theta_1} P_\theta(E) d\mu(\theta), \qtext{for all} E \in \calF_{\calX}. 
	\end{align}
	Then, we have the following lower bound: 
	\begin{align}
		R^*(\Theta, \calW) = \inf_{P_{W|\X}} \sup_{\theta \in \Theta} \mathbb{E}_{\theta}\lb L(\theta, W) \rb \; \geq \;  \omega \lp 1 - TV(P_{\theta_0}, P_\mu) \rp.  \label{eq:point-vs-mixture-1}
	\end{align}
\end{theorem}
\begin{remark}
\label{remark:mixture-justification-1}
	Observe that the only difference from the two-point lower bound we saw in the previous lecture is that $TV(P_{\theta_0}, P_{\theta_1})$ is replaced with $TV(P_{\theta_0}, P_{\mu})$. Since $P_\mu = \mathbb{E}_{\theta \sim \mu}[P_\theta]$, the convexity property of total variation implies that $TV(P_{\theta_0}, P_\mu) \leq \mathbb{E}_{\theta_1 \sim \mu}[TV(P_{\theta_0}, P_{\theta_1} )]$. This fact hints at the use cases of~Theorem~\ref{theorem:point-vs-mixture} to be problems where the total variation~(and other divergences) between the mixture $P_{\mu}$ and $P_{\theta_0}$ can be much smaller than the total variation between $P_{\theta_0}$ and any individual $P_{\theta}$ for $\theta \in \Theta_1$. 
\end{remark}
\emph{Proof of Theorem~\ref{theorem:point-vs-mixture}.} Let $\pi = \frac{1}{2} \lp \delta_{\theta_0} + \mu \rp$ denote a prior distribution over the parameter space. Then, we know that the minimax risk is always lower bounded by any Bayes risk, which implies 
\begin{align}
	\sup_{\theta \in \Theta} R(\theta, P_{W|\X}) \geq R(\pi, P_{W|\X}) = \frac{1}{2}\lp \mathbb{E}_{\theta_0}[L(\theta_0, W)]  +  \mathbb{E}_{\underline{\theta} \sim\mu}[L(\underline{\theta}, W)]\rp
\end{align}
Now, we can expand the two terms in the RHS as follows~(assuming densities $p_\theta$, $p_\mu$): 
\begin{align}
\mathbb{E}_{\theta_0} [L(\theta_0, W)] &= \int_{\calX} \lp \int_{\calW} L(\theta_0, w) p_{W|\X}(w|x) dw \rp p_{\theta_0}(x) =: \int f_0(x) p_{\theta_0}(x) dx, \quad \text{and} \\
\mathbb{E}_{\underline{\theta} \sim\mu}[L(\underline{\theta}, W)] &= \int_{\calX} p_{\mu}(x) \lp \int_{\Theta_1} \frac{p_{\theta}(x)}{p_\mu(x)} \mu(\theta) \lp \int_{\calW} L(\theta, w) p_{W|\X}(w|x)dw \rp d\theta \rp dx \\
&= \int_{\calX} p_{\mu}(x) \lp \int_{\calW} p_{W|\X}(w)  \lp \int_{\Theta_1} L(\theta, w) \frac{p_{\theta}(x)}{p_\mu(x)} \mu(\theta) d\theta \rp dw \rp dx \\
& \coloneqq \int_{\calX} g(x) p_{\mu}(x) dx
\end{align}
Now, observe that $(p_\theta(x) \mu(\theta)/p_\mu(x)) = \mu(\theta|x)$ is the posterior distribution of $\underline{\theta}$, which implies  
\begin{align}
f(x) + g(x) &= \int_{\calW} p_{W|\X}(w|x) \lp  \int_{\Theta_1} \lp L(\theta, w) + L(\theta_0, w)  \rp \mu(\theta|x) d\theta  \rp dw \\
& \geq 2\omega \int_{\calW} p_{W|\X}(w|x) \lp \int_{\Theta_1} \mu(\theta|x) d\theta \rp dw = 2\omega. 
\end{align}
The inequality above relies on the separation assumption. Finally, this implies that 
\begin{align}
\sup_{\theta \in \Theta} R(\theta, P_{W|\X}) \; &\geq \; \frac{1}{2}\lp  \int f(x) p_{\theta_0}(x) dx + \int g(x) p_{\mu}(x) dx \rp \\
& \; \geq\; \frac 1 2 \int \lp f(x) + g(x) \rp  \min\{ p_{\theta_0}(x), \, p_{\mu}(x) \} dx \\
& \geq  \omega \int \min \{ p_{\theta_0}(x), \, p_{\mu}(x) \} dx = \omega (1-TV(P_{\theta_0}, \, P_{\mu})), 
\end{align}
where the last equality uses the fact that $TV(P, Q) = 1 - \int \min\{p, q\}$. 
\hfill \qedsymbol
\begin{remark}
	\label{remark:extension-to-two-mixtures} A close look at the proof suggests that the same argument would have also worked for the case of two mixtures (instead of point-vs-mixture). 
\end{remark}
%%%%%%


\subsection{Divergence Bound}
\label{subsec:divergence-bound}
	To apply Theorem~\ref{theorem:point-vs-mixture} in practice, we first need to fix $(\theta_0, \Theta_1)$, then select an appropriate mixture distribution $\mu$, and finally get an upper bound on $TV(P_{\theta_0}, P_\mu)$. In most cases, it suffices to choose $(\theta_0, \Theta_1)$ to maximize $\omega$~(the separation) while controlling $TV(P_{\theta_0}, P_\mu)$ to a value less than $1/2$. It turns out that for mixtures, working with chi-squared distance is most convenient, as we explain next. 

	For simplicity, throughout we will assume that $P_\theta$ has a density $p_\theta$ with respect to some common dominating measure~(which we simply denote by $dx$). Recall that the chi-squared divergence between any pair $(P,Q)$ with densities $(p,q)$ is defined as 
	\begin{align}
	\chi^2(P\parallel Q) = \int \lp\frac{p(x)}{q(x)} - 1 \rp^2 q(x) dx = \int \frac{\big(p(x)-q(x)\big)^2}{q(x)} dx = \int \frac{p(x)^2}{q(x)} dx -1.  
	\end{align}	
	Our next result shows why this formulation of chi-squared is useful in handling mixtures. 

	\begin{lemma}
		\label{lemma:mixture-chi-squared}
		Assume throughout that $\int p_\theta(x)^2/p_{\theta_0}(x)dx < \infty$ for all $\theta \in \Theta_1$. Then, we have the following: 
		\begin{align}
			1 + \chi^2(P_\mu, P_{\theta_0}) &= \mathbb{E}_{\theta, \theta' \sim \mu}\lb \int \frac{p_{\theta}(x) p_{\theta'}(x)}{p_{\theta_0}(x)} dx \rb = \mathbb{E}_{\theta, \theta' \sim \mu}\lb \int \ell_{\theta, \theta_0}(x) \ell_{\theta', \theta_0}(x) p_{\theta_0}(x)dx \rb \\
			& = \mathbb{E}_{\theta, \theta' \sim \mu} \lb \langle \ell_{\theta, \theta_0}, \ell_{\theta', \theta_0} \rangle_{L^2(P_{\theta_0})} \rb, 
		\end{align}
		where $\ell_{\theta, \theta_0}(x) = p_\theta(x)/p_{\theta_0}(x)$, and we use $\langle \cdot, \cdot \rangle_{L^2(P_{\theta_0})}$ to denote the inner product in the space of square integrable functions (w.r.t. $P_{\theta_0}$).  
	\end{lemma}

	\emph{Proof of Lemma~\ref{lemma:mixture-chi-squared}.}
	We know from the definition of chi-squared divergence that 
	\begin{align}
		1 + \chi^2(P_\mu \parallel P_{\theta_0}) &= \int \frac{p_\mu(x)^2}{p_{\theta_0}(x)} dx = \int \frac{ \lp\int p_\theta(x) d\mu(\theta) \rp \lp\int p_{\theta'}(x) d\mu(\theta') \rp  }{p_{\theta_0(x)}} dx \\
		& = \int d\mu(\theta) \int d\mu(\theta') \int \frac{p_\theta(x) p_{\theta'}(x)}{p_{\theta_0}(x)} dx. 
	\end{align}
This completes the proof.  \hfill \qedsymbol


This result indicates that the chi-squared divergence between a point and a mixture distribution depends on the average ``similarity''~(as measured by the inner product) between two randomly drawn independent distributions according to $\mu$. 

In many applications, we work with \iid observations, and our next result shows the crucial property of tensorization which makes chi-squared divergence the appropriate choice when working with mixtures. 

\begin{lemma}
\label{lemma:chi-squared-tensorization} For any $\theta \in \Theta$, let $P_{\theta}^n$ denote  the $n$-fold product measure, and for some probability measure $\mu$ supported on $\Theta_1$, let $P_\mu^n$ denote $\mathbb{E}_{\underline{\theta} \sim \mu}[P_{\underline{\theta}}^n]$. Then, with $\kappa(\theta, \theta') \coloneqq \langle \ell_{\theta, \theta_0}, \ell_{\theta', \theta_0} \rangle_{L^2(P_{\theta_0})}$, we have the following: 
\begin{align}
1 + \chi^2\lp P_{\mu}^n \parallel P_{\theta_0}^n \rp = \mathbb{E}_{\utheta, \utheta' \sim \mu}\lb \kappa(\utheta, \utheta')^n \rb. 
\end{align}
\end{lemma}
\emph{Proof of Lemma~\ref{lemma:chi-squared-tensorization}.} The proof is a simple consequence of the previous derivation. In particular, from~Lemma~\ref{lemma:mixture-chi-squared}, we know that 
\begin{align}
1 + \chi^2(P_{\mu}^n \parallel P_{\theta_0}^n) = \mathbb{E}_{\utheta, \utheta' \sim \mu}\lb \left \langle \frac{p_{\utheta}^n(x^n)}{p_{\theta_0}^n(x^n)}, \frac{p_{\utheta'}^n(x^n)}{p_{\theta_0}^n(x^n)}\right\rangle_{L^2(P_{\theta_0^n})} \rb. 
\end{align}
Now, on expanding the inner product term, we get 
\begin{align}
\left \langle \frac{p_{\utheta}^n(x^n)}{p_{\theta_0}^n(x^n)}, \frac{p_{\utheta'}^n(x^n)}{p_{\theta_0}^n(x^n)}\right\rangle_{L^2(P_{\theta_0^n})}  &= \int \frac{p_{\utheta}^n(x^n) p_{\utheta'}^n(x^n)}{p_{\theta_0}^n(x^n)} dx^n \\
&= \int \frac{p_{\utheta}(x_1) p_{\utheta'}(x_1)}{p_{\theta_0}(x_1)} dx_1 \ldots \int \frac{p_{\utheta}(x_1) p_{\utheta'}(x_1)}{p_{\theta_0}(x_1)}dx_n  = \kappa(\utheta, \utheta')^n. 
\end{align}
This completes the proof. \hfill \qedsymbol
\begin{remark}
	\label{remark:chi-squared-tensorization}
	The previous two lemmas tell us that the chi-squared divergence between $P_\mu^n = \mathbb{E}_{\utheta \sim \mu}[P_\theta^n]$ and $P_{\theta_0}$ is controlled by the average value of $\kappa(\utheta, \utheta')$; a measure of how similar two randomly drawn product distributions in $\Theta_1$ are. This gives us an indication of the type of problems in which the point-vs-mixture approach is useful: if each $\theta_1$ is such that $P_{\theta_0}$ and $P_{\theta_1}$ are quite distinct from each other, but any two $P_{\theta_1}$ and $P_{\theta_1'}$ are almost ``orthogonal''. In such cases, the two-point method would lead to a suboptimal lower bound~(owing to the large distinctness between $P_{\theta_0}$ and $P_{\theta_1}$), but a mixture method may be more useful~(owing to the almost orthogonality between two randomly drawn elements from $\mu$). 
\end{remark}

\section{Application: Uniformity Testing} 
\label{sec:application-1-uniformity-testing}

Let us consider the hypothesis testing problem within the minimax framework. For some $\{P_\theta: \theta \in \Theta\}$, we are given $n$ \iid observations $\X = X^n = (X_1, \ldots, X_n)$ drawn from an unknown $P_\theta$. Our goal is to test between 
\begin{align}
	H_0: \theta \in \Theta_0, \qtext{versus} H_1: \theta \in \Theta_1, \quad \text{for disjoint } \Theta_0, \Theta_1 \subset \Theta. 
\end{align}
A randomized hypothesis test can be represented by a mapping $\Psi:\calX_n \coloneqq \calX^n \to [0,1]$, with $\Psi(\x)$ denoting the probability of deciding that $H_1$ is true. In other words, the decision space is $\calW = \{0,1\}$, and our decision is $W \sim \Bernoulli(\Psi(X^n))$. Then, the minimax risk with the $0$-$1$ loss is defined as 
\begin{align}
R_n^*(\Theta_0, \Theta_1) = \inf_{\Psi} \sup_{\theta \in \Theta_0 \cup \Theta_1} \mathbb{E}_{\theta}[\boldsymbol{1}_{W \neq h_\theta}] = \inf_{\Psi} \sup_{\theta \in \Theta_0 \cup \Theta_1} \mathbb{P}_{\theta}(W \neq h_\theta), \qtext{where} h_\theta = \boldsymbol{1}_{\theta \in \Theta_1}. 
\end{align}

A simple instance of this problem is for the identity testing for discrete distributions. Assume that $X_1, \ldots, X_n \overset{i.i.d.}{\sim} P_X$ for some distribution supported on a finite alphabet $\calX$ with $|\calX|=k$, and let $U_k$ denote the uniform distribution over $\calX$. Then, for some $\epsilon>0$, consider the problem: 
\begin{align}
H_0: P_X = U_k, \qtext{versus} H_1: \|P_X - U_k\|_1 \geq \epsilon. 
\end{align}
Here, the parameter space is $\Theta = \Delta_k$, with $\Theta_0 = \{\theta_0\}$ and $\Theta_1 = \{\theta: \|\theta_0- \theta\|_1 \geq \epsilon\}$, where $\theta_0 = (1/k, \ldots, 1/k)$. 
This task is called uniformity testing in the theoretical computer science literature. 


\paragraph{Lower Bound via Theorem~\ref{theorem:point-vs-mixture}.} The first step is note that the ``separation condition'' is satisfied with $\omega = 1/2$: for any $\theta \in \Theta_1$, we have 
\begin{align}
L(\theta_0, w) + L(\theta, w) = \boldsymbol{1}_{w=1} + \boldsymbol{1}_{w=0} = 1. 
\end{align}
Hence,~Theorem~\ref{theorem:point-vs-mixture} implies that the minimax risk in this case is lower bounded  by 
\begin{align}
R_n^*(\theta_0, \epsilon) \geq \sup_{\mu} \frac{1}{2}\lp 1 - TV(P_\mu, P_{\theta_0}) \rp,
\end{align}
where $\mu$ is any probability measure of the alternative set. We will now describe Paninski's construction of this mixture. 

Assume that $k$ is even, and pair off the coordinates into $\{(2j-1, 2j): 1 \leq j \leq k/2\}$. let $\vv = (v_1, \ldots, v_{k/2}) \in \{-1, \}^{k/2}$ be drawn \iid from a Rademacher distribution~(i.e., $\pm 1$ w.p. $1/2$ each), and define 
\begin{align}
q_{\vv} \in \Delta_k, \qtext{with}
q_{\vv}[2j-1] = \frac{1+\epsilon v_j}{k}, \qtext{and} 
q_{\vv}[2j] = \frac{1- \epsilon v_j}{k}, \quad \text{for } j \in [k/2]. 
\end{align}
It is easy to verify that each $q_{\vv}$ lies in $\Theta_1$, and the mixture distribution $\mu$ is uniformly distributed over the subset $\{q_{\vv}: \vv \in \{-1, 1\}^{k/2} \} \subset \Theta_1$. Interestingly, we have $\mathbb{E}_{\vv}[q_{\vv}] = p_{\theta_0}$. 
Based on this, we can compute the chi-squared divergence as follows: 
\begin{align}
\kappa(\vv, \vv') &= \sum_{i=1}^k \frac{q_{\vv}[i] q_{\vv'}[i]}{1/k} = k \sum_{j=1}^{k/2} \lp \frac{1 + \epsilon v_j}{k} \frac{1 + \epsilon v'_j}{k} + \frac{1 - \epsilon v_j}{k} \frac{1 - \epsilon v'_j}{k} \rp \\
& = \frac{1}{k} \sum_{j=1}^{k/2} \lp 1 + \epsilon(v_j+v'_j) + \epsilon^2v_jv'_j + 1 - \epsilon(v_j + v'_j) + \epsilon^2 v_j v'_j \rp = 1 + \frac{2 \epsilon^2}{k} \sum_{j=1}^{k/2} v_j v'_j. 
\end{align}
Now, observe that each $s_j \coloneqq v_j v'_j$ is also a Rademacher random variable. Hence, we have 
\begin{align}
1 + \chi^2(P_{\mu}^n \parallel P_{\theta_0}) &= \mathbb{E}_{\vv, \vv'} \lb \lp 1 + \frac{2 \epsilon^2}{k} \sum_{j=1}^{k/2} s_j \rp^n \rb \\
& \leq \mathbb{E}_{\vv, \vv'} \lb \exp \lp \frac{2n \epsilon^2}{k} \sum_{j=1}^{k/2} s_j \rp \rb && (\text{since } 1 + x \leq e^x) \\
& =  \prod_{j=1}^{k/2} \mathbb{E}\lb \exp \lp \frac{2n \epsilon^2}{k}  s_j \rp \rb  = \prod_{j=1}^{k/2} \frac 1 2 \lp e^{2n\epsilon^2/k} + e^{-2n\epsilon^2/k} \rp \\
& \leq  \prod_{j=1}^{k/2} e^{2n^2 \epsilon^4/k^2} && (\text{since } e^x + e^{-x} \leq 2 e^{x^2/2}) \\
& = e^{n^2 \epsilon^4/k}. 
\end{align}

Thus, using the fact that $TV(P, Q) \leq \sqrt{\chi^2(P \parallel Q)/2}$, we get 
\begin{align}
R^*_n(\theta_0, \epsilon) \geq \frac{1}{2}\lp 1 - \sqrt{\frac{e^{n^2 \epsilon^4/k} -1 }{2}} \rp. \label{eq:uniformity-testing-1}
\end{align}

\paragraph{Interpreting the lower bound.} This lower bound can be used to characterize fundamental limits on either the detection boundary, or the sample complexity. In particular, suppose we wish to answer the question: For a fixed $n$, $k$, suppose we have a procedure that can achieve a minimax risk of $r \in (0,1)$. Then, what is the smallest possible value of $\epsilon \equiv \epsilon_{n,k,r}$? To answer this, note that~\eqref{eq:uniformity-testing-1} implies 
\begin{align}
\log (1 + 2(1 + 2r)^2) \leq  {n^2 \epsilon^4/k}  \quad \implies \quad \epsilon_{n, k, r} \geq \frac{c_r k^{1/4}}{\sqrt{n}} \qtext{for} c_r = \lp \log (1 + 2(1 + 2r)^2)\rp^{1/4}. 
\end{align}
This characterizes the \emph{detection boundary}, or a lower limit on the closest alternative that can be distinguished well enough by any test. 
Conversely, we can also characterize the sample complexity,  which is the smallest $n$ for which there exists a procedure with a minimax risk of $r$~(with $\epsilon, k$ fixed). The above equation tells us that 
\begin{align}
n_{\epsilon,k, r} \geq \frac{c_r^2 \sqrt{k}}{\epsilon^2}. 
\end{align}
The key benefit of the two point method is that it can capture the $k$-dependence of the detection boundary / sample complexity. 

\paragraph{Failure of the two-point method.} If we were to use the two-point method, then we have for any $Q$ in the alternative class
\begin{align}
\chi^2(Q^n \parallel P_{\theta_0}^n) = \lp 1 + \chi^2(Q \parallel P_{\theta_0})^n \rp - 1. 
\end{align}
Now, we can show that 
\begin{align}
\inf_{q: \|q-p_{\theta_0}\|_1 \geq \epsilon} \chi^2 (q \parallel p_{\theta_0}) = \epsilon^2, 
\end{align}
achieved at the pmf with equal $\pm \epsilon$  perturbation from the uniform pmf. This gives us 
\begin{align}
\chi^2(Q^n \parallel P_{\theta_0}^n) \leq (1+ \epsilon^2)^n - 1 \leq e^{n \epsilon^2} - 1. 
\end{align}
This will result in 
\begin{align}
\epsilon_{n, k, r} \gtrsim \frac{1}{\sqrt{n}}, \qtext{and} n_{\epsilon, k, r} \gtrsim \frac{1}{\epsilon^2}, 
\end{align}
thus not capturing the $k$ dependence. 


\paragraph{Achievability.} One constructive approach for addressing this task is based on the so-called ``collision statistic'' 
\begin{align}
C_n = \frac{1}{{n \choose 2}}\sum_{i \neq j} \boldsymbol{1}_{X_i = X_j}. 
\end{align}
The idea is that in expectation the number of collisions will be the smallest under the uniform distribution, and hence we can reject the null if $C_n$ is above an appropriately chosen threshold. We will work out the details in Homework 2. 


% 
% \bibliographystyle{abbrvnat}           % if you need a bibliography
% \bibliography{../ref}                % assuming yours is named ref.bib


\end{document}