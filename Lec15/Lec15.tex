\documentclass[12pt]{article}
\usepackage[english]{babel}
\usepackage[utf8x]{inputenc}
\usepackage[T1]{fontenc}
\usepackage{../scribe}
\usepackage{listings}

\usepackage{hyperref}
% \usepackage{cleveref}
\usepackage{xcolor}
\usepackage{natbib}
\usepackage{tikz}
\renewcommand{\arraystretch}{1.5}
\usepackage{nicefrac}
% \newcommand{\thetahat}{\widehat{\theta}}
% \Scribe{Shubhanshu Shekhar}
\Lecturer{Shubhanshu Shekhar}
\LectureNumber{15}
\LectureDate{21st October, 2025}
\LectureTitle{Sequential Inference Problems}

\lstset{style=mystyle}
% \usepackage{tikz}
\usepackage{comment}
% \usepackage{pgfplots}
% \usepackage{enumitem}
\usepackage{amsmath,amssymb,amsthm,amsfonts,latexsym,bbm,xspace,graphicx,float,mathtools,mathdots}
\usepackage{braket,caption,ellipsis,xcolor,textcomp}
\usepackage{combelow} %for you, Rusins Freivalds
% \usepackage[backref,colorlinks,citecolor=blue,bookmarks=true]{hyperref}
\usepackage{booktabs}
\usepackage{hyperref}
% \usepackage[capitalize,noabbrev]{cleveref}
% \crefname{ineq}{inequality}{inequalities}
% \creflabelformat{ineq}{#2{\upshape(#1)}#3}

% removed the next two oas it is preloaded in neurips_2022.sty
\usepackage[letterpaper,margin=1in]{geometry}
\usepackage{enumitem} 
%\usepackage[protrusion=true,expansion=true]{microtype}   % slows compilation a loeve


% environments
% \newtheorem{theorem}{Theorem}[section]

\newtheorem{question}[theorem]{Question}
\newtheorem{answer}[theorem]{Answer}
\newtheorem{hypothesis}[theorem]{Hypothesis}
\newtheorem{assumption}[theorem]{Assumption}
\newtheorem{result}[theorem]{Main Result}
\newtheorem{problem}[theorem]{Problem}



\theoremstyle{definition}
\newtheorem{examples}[theorem]{Example}



\newcommand{\dist}{\textnormal{dist}}
\newcommand{\vol}{\textnormal{vol}}
\newcommand{\spn}{\textnormal{span}}
\newcommand{\supp}{\textnormal{supp}}
\newcommand{\tr}{\operatorname{tr}}
\newcommand{\codim}{\operatorname{codim}}
\newcommand{\diag}{\operatorname{diag}}

\newcommand{\bbone}{\mathbbm 1}
\newcommand{\Id}{\bbone}

% complexity classes
\newcommand{\PTIME}{\mathsf{P}}
\newcommand{\LOGSPACE}{\mathsf{L}}
\newcommand{\ZPP}{\mathsf{ZPP}}
\newcommand{\RP}{\mathsf{RP}}
\newcommand{\TC}{\mathsf{TC}}
\newcommand{\AC}{\mathsf{AC}}
\newcommand{\SC}{\mathsf{SC}}
\newcommand{\SZK}{\mathsf{SZK}}
\newcommand{\AM}{\mathsf{AM}}
\newcommand{\IP}{\mathsf{IP}}
\newcommand{\PSPACE}{\mathsf{PSPACE}}
\newcommand{\EXP}{\mathsf{EXP}}
\newcommand{\MIP}{\mathsf{MIP}}
\newcommand{\NEXP}{\mathsf{NEXP}}
\newcommand{\BQP}{\mathsf{BQP}}

% short forms
\newcommand{\dleta}{\delta}
\newcommand{\simga}{\sigma}
\newcommand{\vphi}{\varphi}
\newcommand{\ol}[1]{\overline{#1}}
\newcommand{\ul}[1]{\underline{#1}}
\newcommand{\ot}{\otimes}
\newcommand{\zo}{\{0,1\}}
\newcommand{\co}{:}                       %\newcommand{\co}{\colon}
\newcommand{\bdry}{\partial}
\newcommand{\grad}{\nabla}
\newcommand{\symmdiff}{\triangle} \newcommand{\symdiff}{\symmdiff}


% calligraphic letters
% bold
\newcommand{\bone}{\boldsymbol{1}}
\newcommand{\bbeta}{\boldsymbol{\beta}}
\newcommand{\bdelta}{\boldsymbol{\delta}}
\newcommand{\bepsilon}{\boldsymbol{\epsilon}}
\newcommand{\blambda}{\boldsymbol{\lambda}}
\newcommand{\bomega}{\boldsymbol{\omega}}
\newcommand{\bpi}{\boldsymbol{\pi}}
\newcommand{\bphi}{\boldsymbol{\phi}}
\newcommand{\bvphi}{\boldsymbol{\varphi}}
\newcommand{\bpsi}{\boldsymbol{\psi}}
\newcommand{\bsigma}{\boldsymbol{\sigma}}
\newcommand{\btheta}{\boldsymbol{\theta}}
\newcommand{\btau}{\boldsymbol{\tau}}
\newcommand{\ba}{\boldsymbol{a}}
\newcommand{\bb}{\boldsymbol{b}}
\newcommand{\bc}{\boldsymbol{c}}
\newcommand{\bd}{\boldsymbol{d}}
\newcommand{\be}{\boldsymbol{e}}
\newcommand{\boldf}{\boldsymbol{f}}
\newcommand{\bg}{\boldsymbol{g}}
\newcommand{\bh}{\boldsymbol{h}}
\newcommand{\bi}{\boldsymbol{i}}
\newcommand{\bj}{\boldsymbol{j}}
\newcommand{\bk}{\boldsymbol{k}}
\newcommand{\bell}{\boldsymbol{\ell}}
\newcommand{\bp}{\boldsymbol{p}}
\newcommand{\bq}{\boldsymbol{q}}
\newcommand{\br}{\boldsymbol{r}}
\newcommand{\bs}{\boldsymbol{s}}
\newcommand{\bt}{\boldsymbol{t}}
\newcommand{\bu}{\boldsymbol{u}}
\newcommand{\bv}{\boldsymbol{v}}
\newcommand{\bw}{\boldsymbol{w}}
\newcommand{\bx}{{\boldsymbol{x}}}
\newcommand{\by}{\boldsymbol{y}}
\newcommand{\bz}{\boldsymbol{z}}
\newcommand{\bA}{\boldsymbol{A}}
\newcommand{\bB}{\boldsymbol{B}}
\newcommand{\bC}{\boldsymbol{C}}
\newcommand{\bD}{\boldsymbol{D}}
\newcommand{\bE}{\boldsymbol{E}}
\newcommand{\bF}{\boldsymbol{F}}
\newcommand{\bG}{\boldsymbol{G}}
\newcommand{\bH}{\boldsymbol{H}}
\newcommand{\bI}{\boldsymbol{I}}
\newcommand{\bJ}{\boldsymbol{J}}
\newcommand{\bL}{\boldsymbol{L}}
\newcommand{\bM}{\boldsymbol{M}}
\newcommand{\bP}{\boldsymbol{P}}
\newcommand{\bR}{\boldsymbol{R}}
\newcommand{\bS}{\boldsymbol{S}}
\newcommand{\bT}{\boldsymbol{T}}
\newcommand{\bU}{\boldsymbol{U}}
\newcommand{\bV}{\boldsymbol{V}}
\newcommand{\bW}{\boldsymbol{W}}
\newcommand{\bX}{\boldsymbol{X}}
\newcommand{\bY}{\boldsymbol{Y}}
\newcommand{\bZ}{\boldsymbol{Z}}

% bold calligraphic
\newcommand{\bcalG}{\boldsymbol{\calG}} \newcommand{\calbG}{\bcalG}
\newcommand{\bcalX}{\boldsymbol{\calX}} \newcommand{\calbX}{\bcalX}
\newcommand{\bcalY}{\boldsymbol{\calY}} \newcommand{\calbY}{\bcalY}
\newcommand{\bcalZ}{\boldsymbol{\calZ}} \newcommand{\calbZ}{\bcalZ}

% left-right wrappers
\DeclarePairedDelimiter\parens{\lparen}{\rparen}
% words -- there's supposed to be something better than xspace though
\newcommand{\Hastad}{H{\aa}stad\xspace}
\newcommand{\Holder}{H{\"o}lder\xspace}
\newcommand{\ErdosRenyi}{Erd\H{o}s--R\'{e}nyi\xspace}

%%%%%%%%%%%%% hyphenation %%%%%%%%%%%%%
\hyphenation{Naka-shima}


%%%%%%%%%%%% document-writing macros %%%%%%%%%%%%

%\newcommand{\ignore}[1]{*\marginpar{\tiny \em ignored part}}



% \newcommand{\defined}{\coloneqq}
\usepackage{autonum}


% 
\usepackage{natbib}
\setcitestyle{authoryear,open={(},close={)}} %Citation-related commands






% % to avoid loading the natbib package, add option nonatbib:
% % \usepackage[nonatbib]{neurips_2020}
% \usepackage[utf8]{inputenc} % allow utf-8 input
% \usepackage[T1]{fontenc}    % use 8-bit T1 fonts
% \usepackage{hyperref}       % hyperlinks
% \usepackage{url}            % simple URL typesetting
% \usepackage{booktabs}       % professional-quality tables
% \usepackage{amsfonts}       % blackboard math symbols
% \usepackage{nicefrac}       % compact symbols for 1/2, etc.
% \usepackage{microtype}      % microtypography
% \usepackage{mathtools}




% % --------------------------------
% % Additional packages and commands
% \usepackage{amsmath,amsfonts,amsthm,amssymb}
% \usepackage[]{xcolor}
% \usepackage[]{natbib}

% \usepackage{cleveref}       % hyperlinks

% \newcommand{\RR}{\mathbb{R}}
\newcommand{\NN}{\mathbb{N}}
\newcommand{\EE}{\mathbb{E}}
% \newcommand{\Xcal}{\mathcal{X}}
% \newcommand{\Ycal}{\mathcal{Y}}
% \newcommand{\Dcal}{\mathcal{D}}
% \newcommand{\Lcal}{\mathcal{L}}
% \newcommand{\Ncal}{\mathcal{N}}
% \newcommand{\Fcal}{\mathcal{F}}
% \newcommand{\GP}{\textnormal{GP}}
% \newcommand{\williex}[1]{\textcolor{red}{[#1]}}
\DeclareMathOperator*{\argmin}{argmin}
\DeclareMathOperator*{\argmax}{argmax}
\DeclareMathOperator*{\esssup}{esssup}

% \newtheorem{lemma}{Lemma}
% \newtheorem{theorem}{Theorem}
% \newtheorem{proposition}{Proposition}
% \newtheorem{corollary}{Corollary}

% \theoremstyle{definition}
% \newtheorem{definition}{Definition}
% \newtheorem{remark}{Remark}
% \newtheorem{example}{Example}
% \newtheorem{assumption}{Assumption}



% \usepackage{graphicx} 
% \usepackage{caption}
% \usepackage{subcaption}
% \usepackage{enumitem}
% % --------------------------------
% \usepackage{xspace}

\newcommand{\iid}{\text{i.i.d.}\xspace}
\newcommand{\prob}{\mathbb{P}}
\newcommand{\expec}{\mathbb{E}}
\newcommand{\simiid}{\stackrel{i.i.d.}{\sim}}


%=======================================
\newcommand{\mc}[1]{\mathcal{#1}}
\newcommand{\mbb}[1]{\mathbb{#1}}
\newcommand{\lb}{\left[}
\newcommand{\rb}{\right]}
\newcommand{\lp}{\left(}
\newcommand{\rp}{\right)}
%=======================================


%=======================================
\newcommand{\defined}{\coloneqq}


%=======================================
\newcommand{\arl}{\text{ARL}\xspace} %average run length 
\newcommand{\delay}{D\xspace} % detection delay
\newcommand{\hatT}{\widehat{T}} %estimated change point 
\newcommand{\hatepsilon}{\widehat{\epsilon}} % estimated change magnitude 
\newcommand{\barB}{\bar{B}} % time-reversed confidence sequence
\newcommand{\cusum}{\text{CuSum}\xspace}

\newcommand{\backwardtest}{\tau^{\text{back}}}
\newcommand{\dkl}{D_{\text{KL}}}
\newcommand{\dks}{d_{\text{KS}}}
\newcommand{\dmmd}{d_{\text{MMD}}}
\newcommand{\red}[1]{\textcolor{red}{#1}}
\newcommand{\blue}[1]{\textcolor{blue}{#1}}
\newcommand{\thetahat}{\widehat{\theta}}
\newcommand{\Xhat}{\widehat{X}}
\newcommand{\muhat}{\widehat{\mu}}
\newcommand{\sigmahat}{\widehat{\sigma}}
\newcommand{\sigmatilde}{\widetilde{\sigma}}
\newcommand{\Phat}{\widehat{P}}
\newcommand{\Qhat}{\widehat{Q}}
\newcommand{\fcsdetector}{\texttt{FCS-Detector}\xspace}
\newcommand{\bcsdetector}{\texttt{BCS-Detector}\xspace}
\newcommand{\rfcsdetector}{\texttt{RCS-Detector}\xspace}
%%% filtration 
\newcommand{\filtration}{\mathcal{F}}

\newcommand{\uniform}{\mathrm{Uniform}}
\newcommand{\mse}{\mathrm{MSE}}
\newcommand{\power}{\mathrm{Power}}
\newcommand{\Ytilde}{\widetilde{Y}}
\newcommand{\mhat}{\widehat{m}}
\newcommand{\tauhat}{\widehat{\tau}}
\newcommand{\thetatilde}{\widetilde{\theta}}
\newcommand{\qtilde}{\widetilde{q}}
\newcommand{\Xtilde}{\widetilde{X}}
\newcommand{\nptest}{\Psi^*_{\mathrm{NP}}}
\newcommand{\umptest}{\psi^*_{\mathrm{UMP}}}
\newcommand{\RTY}{R_{\Theta, Y}}
\newcommand{\PTY}{P_{\Theta, Y}}
\newcommand{\RR}{\boldsymbol{R}}
\newcommand{\Hab}{\mathcal{H}_a^b} 
\newcommand{\Yab}{Y_a^b}
\newcommand{\boldw}{\boldsymbol{w}}
\newcommand{\lpost}{L_{\text{post}}}
\newcommand{\cramer}{Cram\'er}
\newcommand{\Beta}{\mathrm{Beta}}
\newcommand{\Bernoulli}{\mathrm{Bernoulli}}
\newcommand{\Binomial}{\mathrm{Binomial}}
\newcommand{\GammaDist}{\mathrm{Gamma}}
\newcommand{\data}{X=(Y_1, \ldots, Y_n)}
\newcommand{\convdist}{\stackrel{d}{\rightarrow}}
\newcommand{\convprob}{\stackrel{p}{\rightarrow}}
\newcommand{\typicalset}[1]{{#1}_\epsilon^{(n)}}
\newcommand{\vhat}{\hat{\boldsymbol{v}}}
\newcommand{\boldX}{\boldsymbol{X}}

\newcommand{\boldtheta}{\boldsymbol{\theta}}


\begin{document}
	\MakeScribeTop
With this lecture, we begin the discussion of the topic of \emph{sequential anytime-valid inference}. This field of study was initiated by Robbins and collaborators in the 1960s-70s, with key contributions from Lai, Seigmund, Darling, and others. Over the last decade there has been a resurgence  of interest in this area, partially driven by the importance of efficient large-scale A/B testing done in the industry~\citep{johari2015always}, and partially by some theoretical and conceptual advances in the foundations in statistics and probability~\citep{shafer2019game}.  A summary of the recent advances in this area can be found in the survey paper by~\citet{ramdas2023game}. 

In this lecture, we will introduce the three main inference tasks within this framework: power-one testing, estimation via confidence sequences, and sequential change detection. We will see a duality between the first two tasks, and then discuss a result by~\citet{lorden1971procedures} that establishes a reduction from the problem of change detection to power-one testing. 


\section{Motivation for Designing Sequential Procedures}
\label{sec:motivation-seq-inference}
In a typical fixed sample size inference problem, we assume that we have access to a dataset $X^n = (X_1, \ldots, X_n) \in \calX^n$ drawn \iid~(for simplicity) from a distribution $P_\theta$ with $\theta \in \Theta$, and our goal is to make a decision about the unknown parameter $\theta$. Here the sample-size $n$ is a quantity that is decided before the experiment or data-collection. In contrast, for sequential procedures, the sample size itself is a random variable, and the process of data-collection and inference are tightly coupled.  The motivation of designing sequential procedures are usually for two reasons: (i) sequential methods can lead to some benefits over their fixed sample size~(FSS) counterparts, and (ii) there exist problems which can only be solved sequentially. We illustrate both situations through examples next. 

\paragraph{Example 1.} Suppose $\calX = \{0,1\}$ and $P_\theta = \Bernoulli(\theta)$ with $\theta \in \Theta = [0,1]$. Consider the hypothesis testing problem 
\begin{align}
    H_0: \theta \in \Theta_0 = [0, \theta_0], \qtext{versus} H_1: \theta \in \Theta_1 = (\theta_0, 1]. 
\end{align}
In the FSS setting, we have $X^n \simiid P_\theta$, and we wish to design a deterministic test $\Psi:\calX^n \to \{0,1\}$ which satisfies $\mathbb{P}_{H_0}(\Psi(X^n)=1) \leq \alpha$ for a prespecified $\alpha$. A natural test in this setting is the likelihood ratio test that takes the form 
\begin{align}
    \Psi(x^n) = \begin{cases}
        1, & \text{ if } S_n = \sum_{i=1}^n X_i \geq t_\alpha, \\
        0, & \text{ otherwise}, 
    \end{cases}
\end{align}
where  $t_\alpha = \min \{m \in \{0, \ldots, n\}: \mathbb{P}_{\theta_0}(S_n \geq m) \leq \alpha\}$. Now, suppose that instead of working with the entire dataset $X^n$ at once, we observed the samples one at a time, and we define a stopping time $\tau$ and a hypothesis test $\Psi_\tau$ as  
\begin{align}
    \tau = \min \{i \leq n: S_i \geq t_\alpha\}, \quad \Psi_\tau = \begin{cases}
        1,& \text{ if } \tau \leq n, \\
        0, & \text{ otherwise}.  
    \end{cases}
\end{align}
It is easy to observe that 
\begin{align}
    \mathbb{P}_{H_0}(\Psi_\tau = 1) = \mathbb{P}_{H_0}(\Psi(X^n)=1), \qtext{and}  \tau \stackrel{a.s.}{\leq} n, 
\end{align}
by construction. Hence, $\Psi_\tau$ has the same significance level as the FSS test $\Psi$, but it requires a random number $\tau \leq n$ of observations; that is, it allows  ``early stopping''.  Thus, the sequential procedure $\Psi_\tau$ is a strict improvement over its FSS analog $\Psi$. In general, sequential methods do not result in such ``pathwise'' improvement; this is a special case due to the monotonicity of $S_n$ and the one-sided nature of the testing problem. 



\paragraph{Example 2: Fixed-width estimation with unknown variance.} Next, we consider a simple problem that clearly illustrates a situation in which non-sequential solutions are impossible. Suppose $X^n = (X_1, \ldots, X_n)$ is drawn \iid from $N(\theta, \sigma^2)$ with both $(\theta, \sigma^2)$ unknown, and fix an $\epsilon >0$ and $\alpha \in (0,1)$. Then, does their exist an $n \geq 1$ and an estimator $\muhat_n \equiv \muhat_n(X^n)$, such that we have 
\begin{align}
    \inf_{\mu\in \mathbb{R}, \sigma^2 > 0} \; \mathbb{P}_{\mu, \sigma^2} \lp \mu \in [\muhat_n - \epsilon,  \muhat_n + \epsilon] \rp \geq 1-\alpha?  \label{eq:fixed-width-estimation}
\end{align}
In other words, can we construct a ``fixed-width'' estimate of Gaussian mean with unknown variance that is uniformly valid  at level $1-\alpha$? It turns out that it is not too difficult to show that such an estimator cannot exist~\citep{dantzig1940non}, and we can establish this with a simple application of LeCam's two-point method due to the fact that $\sigma^2$ can be arbitrarily large.  

\begin{proposition}
    \label{prop:impossibility-fixed-width-estimation} For a fixed $\alpha \in (0,1/2)$ and $\epsilon>0$, there exists no $n$ and estimator $\muhat_n$ for which~\eqref{eq:fixed-width-estimation} holds. 
\end{proposition}
\begin{proof}
    Suppose there exists an $n \geq 1$ and an estimator $\muhat_n$ for which~\eqref{eq:fixed-width-estimation} holds. Now, consider two mean values $\mu_1 = -2\epsilon$ and $\mu_2 = 2\epsilon$, and define the sets 
    \begin{align}
        E_j = \{x^n \in \calX^n: |\muhat_n(x^n) - \mu_j| \leq \epsilon \}, \qtext{for} j= 1,2. 
    \end{align}
    Then, by~\eqref{eq:fixed-width-estimation}, we know that 
    \begin{align}
        \inf_{\sigma^2>0} \mathbb{P}_{\mu_j, \sigma^2}(X^n \in E_j) \geq 1-\alpha, \qtext{for} j= 1, 2.
    \end{align}
    Note that since $|\mu_1 - \mu_2| = 4 \epsilon$,  $E_1$ and $E_2$ are disjoint subsets of $\calX^n$. 
    Hence, $E_2 \subset E_1^c$, which leads to the following 
    \begin{align}
        \mathbb{P}_{\mu_1, \sigma^2}(X^n \in E_1) - \mathbb{P}_{\mu_2, \sigma^2}(X^n \in E_1) &= \mathbb{P}_{\mu_1, \sigma^2}(X^n \in E_1) - 1 +  \mathbb{P}_{\mu_2, \sigma^2}(X^n \in E_1^c) \\
        & \geq \mathbb{P}_{\mu_1, \sigma^2}(X^n \in E_1) - 1 +  \mathbb{P}_{\mu_2, \sigma^2}(X^n \in E_2). 
    \end{align}
    On taking $1$ to the other side, and by using the definition of total variation and~\eqref{eq:fixed-width-estimation}, we get 
    \begin{align}
        TV(\mathbb{P}_{\mu_1, \sigma^2}, \mathbb{P}_{\mu_2, \sigma^2}) + 1 \geq \mathbb{P}_{\mu_1, \sigma^2}(X^n \in E_1)  +  \mathbb{P}_{\mu_2, \sigma^2}(X^n \in E_2) \geq 2 - 2\alpha. 
    \end{align}
    Next, by using Pinsker's inequality, we get that $TV(\mathbb{P}_{\mu_1, \sigma^2}, \mathbb{P}_{\mu_2, \sigma^2}) \leq \sqrt{n}\epsilon/\sigma$, which implies 
    \begin{align}
        \frac{\epsilon}{\sigma} \sqrt{n} \geq 1 - 2\alpha, \qtext{which is violoated for all} \sigma > \frac{\epsilon \sqrt{n}}{1 - 2\alpha}.
    \end{align}
    This completes the proof. 
\end{proof}
Thus, we cannot construct an estimator which can achieve a fixed width $\epsilon$ uniformly over all $\sigma^2$ for any finite $n$: there always will be some problems with large enough $\sigma^2$ to make this impossible. However, sequential methods are naturally suited to this task, since we can hope to construct procedures that automatically adapt the (random) sample-size $\tau$ to the unknown variance, with $\tau$ taking larger values for larger $\sigma^2$ to attain the same $\epsilon$ width. This was first proved by~\citet{lai1976confidence}, who used a generalized likelihood ratio test using improper priors in order to design the test.  We state a version of this result described by~\citet[Theorem 4.1]{wang2025anytime}. 
\begin{proposition}[Theorem 4.1 of~\citet{wang2025anytime}]
    \label{prop:wang-ramdas-t-test} Fix an $m \geq 2$, and for a given $\alpha \in (0, 1)$, let $a$ denote a solution of the equation 
    \begin{align}
        2(1-F_{m-1}(a) + a f_{m-1}(a)) = \alpha, 
    \end{align}
    where $F_{m-1}$ and $f_{m-1}$ denote the CDF and PDF (resp.) of the $t$-distribution with $m-1$ degrees of freedom. Then, we have the following: 
    \begin{align}
        &\mathbb{P}_{\mu, \sigma^2}\lp \forall n \geq m: \mu \in  \lb \muhat_n - A_n s_n, \, \muhat_n + A_n s_n \rb \rp \geq 1-\alpha, \\ 
        \text{where} \quad &A_n =  \sqrt{\lp \frac{1}{m}\lp 1 + \frac{a^2}{m-1}\rp^{m}n \rp^{1/n} - 1}, \; \text{and} \; 
        s_n = \frac{1}n \sum_{i=1}^n \lp X_i - \muhat_n \rp^2. 
    \end{align}
\end{proposition}
This result guarantees the existence of a so-called \emph{confidence sequence} that we will describe later in~Section\ref{sec:confidence-sequences}, and it can be used to define the random stopping time $\tau$ as 
\begin{align}
    \tau = \inf \{n \geq 1: A_n s_n \leq \epsilon\}, \qtext{with} \inf \emptyset = + \infty. 
\end{align}
It is not too difficult to show that for every $(\mu, \sigma^2)$ pair, we have $\mathbb{P}_{\mu, \sigma^2}(\tau< \infty) = 1$. Hence, the sequential procedure $I_\tau = [\muhat_\tau \pm A_\tau s_\tau]$ satisfies the required condition~\eqref{eq:fixed-width-estimation}. 

\section{Sequential ``Power-one'' Tests}
For the remainder of this lecture, we will work with some measurable space $(\calX^\infty, \calF)$, containing an indexed class of probability measures $\{\mathbb{P}_\theta: \theta \in \Theta\}$.Let $\Theta_0, \Theta_1$ denote two disjoint subsets of the parameter set $\Theta$, and given a stream of $\calX$-valued observations  $\{X_i: i \geq 1\} \sim \mathbb{P}_{\theta}$, consider the hypothesis testing problem 
\begin{align}
    H_0: \theta \in \Theta_0, \qtext{versus} H_1: \theta \in \Theta_1. \label{eq:general-hypothesis-teting-problem}
\end{align}
Our goal is to construct a level-$\alpha$ power-one sequential test for this problem, that we formally define next. 
\begin{definition}
\label{def:power-one-test}  Let $\{\calF_n \subset \calF: n \geq 0\}$ denote a filtration such that $\sigma(X_1,\ldots, X_n) = \calF_n$ for all $n \geq 1$. Then, a level-$\alpha$ power-one test for the problem~\eqref{eq:general-hypothesis-teting-problem} is a pair $(\tau, (\phi_k)_{k \in \mathbb{N}})$, such that $\tau$ is a $\mathbb{N} \cup \{\infty\}$ valued stopping time, and $\phi_k:\calX^k \to \{0,1\}$ is a $\calF_k$-measurable ``decision rule'' for all $k \in \mathbb{N}$, such that the following conditions hold: 
\begin{align}
    &\sup_{\theta \in \Theta_0}\mathbb{P}_{\theta}\lp \phi_{\tau}(X^\tau) = 1 \rp  \leq \alpha, && (\text{Level-$\alpha$ Property}) \\
    &\inf_{\theta \in \Theta_1}\mathbb{P}_{\theta}\lp \phi_{\tau}(X^\tau) = 1 \rp  = 1. && (\text{Power-one Property})
\end{align}
In many instances we simply have $\phi_\tau = \boldsymbol{1}_{\tau < \infty}$; the decision is to always reject the null, and we will often follow this convention and simply refer to the stopping time $\tau$ as the level-$\alpha$ power-one test.  
\end{definition}

\begin{example}
    \label{example:power-one-test-1} Consider a simple null $\Theta_0 = \{\theta_0\}$ and simple alternative $\Theta_1 = \{\theta_1\}$, and suppose each $\mathbb{P}_\theta$ is an infinite product of $P_\theta$ with density $p_\theta$~(w.r.t. some common dominating measure). Then, given a stream $\{X_i: i \geq 1\}$ drawn from an unknown $\mathbb{P}_\theta = \otimes_{i=1}^\infty p_\theta$, we can define a level-$\alpha$ power-one stopping time $\tau$ as 
    \begin{align}
        \tau = \inf \{n \geq 1: W_n \geq 1/\alpha\}, \qtext{with} W_0=1, \; W_n = W_{n-1} \times \frac{p_{\theta_1}(X_n)}{p_{\theta_0}(X_n)}. 
    \end{align}
    The process $\{W_n: n \geq 1\}$ is the likelihood ratio process, and as we will see that is a \emph{nonnegative martingale} with an initial value $1$ under the null. That is $\mathbb{E}[W_n \mid \calF_{n-1}] \stackrel{a.s.}{=} W_{n-1}$, and hence such a process satisfies a time-uniform variant of Markov's inequality, called Ville's inequality, that says that for any $a >0$, we have 
    \begin{align}
        \mathbb{P}_{\theta_0} \lp \exists n \geq  1: W_n \geq a \rp \leq \frac{1}{a}. 
    \end{align}
    This immediately implies the ``level-$\alpha$ property'' of $\tau$ by selecting $a \leftarrow 1/\alpha$. Furthermore, when the alternative is true, then we can show that 
    \begin{align}
        \mathbb{E}[\tau] = \calO\lp \frac{\log (1/\alpha)}{\dkl(P_{\theta_1} \parallel P_{\theta_0}) } \rp, 
    \end{align}
    which is finite whenever $\dkl(P_{\theta_1} \parallel P_{\theta_0})>0$, and hence also satisfies the power-one property.
\end{example}

The above example illustrates the general steps~(in the simplest setting) we will follow in constructing power-one tests, and we will explore this in more details in the next lecture. 

\section{Estimation via Confidence Sequences}
\label{sec:confidence-sequences}

\begin{definition}
    \label{def:conf-seqs}  Let $\{X_n: n \geq 1\} \sim \mathbb{P}_\theta$ denote an infinite sequence of observations in $ \calX^\infty$ for some unknown $\theta \in \Theta$. Then, a sequence of possibly random subsets $\{C_n: n \geq 0\}$ is said to form a level-$(1-\alpha)$ confidence sequence for $\theta$ if each $C_n \equiv C_n(X^n)$ is constructed based on $(X_1, \ldots, X_n)$, and 
    \begin{align}
        \mathbb{P}_{\theta}\lp  \forall n \geq 1: \theta \in C_n \rp \geq 1- \alpha, \quad \iff \quad \mathbb{P}_{\theta}\lp  \exists n \geq 1: \theta \not \in C_n \rp \leq \alpha.
    \end{align}
    That is, a confidence sequences~(CSs) satisfy a \emph{uniform validity} property~(the quantifier $\forall n \geq 1$ is inside the probability statement). 
\end{definition}

\begin{remark}
    A natural consequence of the above definition is that if $N$ is a random stopping time w.r.t. to the natural filtration $\{\calF_n: n \geq 1\}$, then we also have 
    \begin{align}
        \mathbb{P}_{\theta}\lp \theta \in C_N \rp \geq 1-\alpha. 
    \end{align}
    In other words, confidence sequences retain their validity at data-driven stopping rules. This makes confidence sequences valid under ``peeking''. 
\end{remark}

\begin{remark}
    Recall that a level-$(1-\alpha)$ confidence interval~(CI) for $\theta$ constructed using $X^n$ is any set $C_n \subset \Theta$ such that 
    \begin{align}
        \forall n\geq 1: \quad \mathbb{P}(\theta \in C_n) \geq 1-\alpha. 
    \end{align}
    This illustrates the crucial difference between CIs and CSs: the quantifier ``$\forall n \geq 1$'' is outside the probability statement in CIs, and hence their validity guarantee is not time-uniform (it holds only for a fixed $n$). 
\end{remark}

\begin{remark}
    \label{remark:fixed-width-estimation-CSs} Due to the uniform validity property of CSs, it is without loss of generality to assume that $C_n \subset C_m$ for all $n \geq m$~(if not, we can simply take the running intersection without violating the validity properties). In most cases, reasonable CSs satisfy $|C_n| \to 0$ as $n\to \infty$ for appropriate notions of size of the sets. If $\calX=\mathbb{R}$, then this suggests that CSs are an appropriate tool for ``fixed width estimation'' problem of Section~\ref{sec:motivation-seq-inference}. More specifically, if we wish to construct an $\epsilon$-width estimate  of a parameter $\theta$, we can do that by using $C_\tau$ with $\tau = \inf \{n \geq 1: |C_n| \leq \epsilon\}$.  
\end{remark}

\paragraph{Duality between CSs and power-one tests.} As in the fixed sample size case, there exists a natural duality between power-one tests and confidence sequences: 
\begin{itemize}
    \item Consider a testing problem in which given $\{X_i: i \geq 1\} \sim \mathbb{P}$s, we wish the test the null $H_\theta: \mathbb{P}=\mathbb{P}_\theta$ against the alternative $H_{\theta, 1}: \mathbb{P}\neq\mathbb{P}_\theta$. Now, suppose we know how to construct a level-$\alpha$ test $\tau_\theta$ for every such null; that is, $\mathbb{P}_{\theta}(\tau_\theta < \infty) \leq \alpha$. Then, we can use these to design a confidence sequence $\{C_n: n \geq 0\}$ with $C_0=\Theta$, and 
    \begin{align}
        C_n = \{\theta \in \Theta: \tau_\theta > n\}, \qtext{for} n\geq 1. 
    \end{align}
    In other words, using the observations $\{X_i: i \geq 1\}$, we run in parallel a continuum of tests $\{\tau_\theta: \theta \in \Theta\}$, and our confidence set at time $n$ is the collection of all parameters $\theta$ for which we do not have enough evidence yet to reject the (corresponding) null $H_\theta$. It is easy to verify that if $\theta^*$ is the true parameter, then 
    \begin{align}
        \mathbb{P}_{\theta^*}\lp \exists n \geq 1: \theta^* \not \in C_n \rp = \mathbb{P}_{\theta^*}\lp \exists n \geq 1: \tau_{\theta^*} \leq n \rp  =  \mathbb{P}_{\theta^*}\lp  \tau_{\theta^*}  < \infty \rp   \leq \alpha. 
    \end{align}
    \item Now, suppose $\{X_i: i \geq 1\} \sim \mathbb{P}_{\theta}$, and we want to test the hypothesis $H_{\theta^*}: \theta = \theta^*$. If we know how to construct a  level-$(1-\alpha)$ CS $\{C_n: n \geq 1\}$, we can define a stopping time 
    \begin{align}
        \tau = \inf \{n \geq 1: \theta^* \not \in C_n\} \; \implies \;  \mathbb{P}_{\theta^*}\lp \tau < \infty \rp = \mathbb{P}_{\theta^*}\lp \exists n \geq 1: \theta^* \not \in C_n \rp \leq \alpha. 
    \end{align}
    The power-one property will require that the width of the set $|C_n|$ converges to $0$ almost surely. 
\end{itemize}

\section{Sequential Change Detection}

This is the third main inference task we will study in this course. In this problem, we again assume we have a stream of observations $\{X_i: i \geq 1\}$, and there is an unknown time $T \in \mathbb{N} \cup \{\infty\}$ at which the distribution changes from $\mathbb{P}_{\theta_0}$ to $\mathbb{P}_{\theta_1}$. In general, we assume that $\theta_0 \in \Theta_0$ and $\theta_1 \in \Theta_1$ for some disjoint classes $\Theta_0, \Theta_1 \subset \Theta$, and our goal is to design a stopping time $\tau$ that satisfies the following: 
\begin{align}
    &\mathrm{minimize} \; D(\tau, \Theta_0, \Theta_1), \; \text{subject to} \; \mathrm{ARL}(\tau, \Theta_0) \geq 1/\alpha, \\
    \text{where} \quad 
    &\mathrm{ARL}(\tau, \Theta_0) = \inf_{\theta_0 \in \Theta_1} \mathbb{E}_{\theta_0}[\tau], \qtext{and} D(\tau, \Theta_0, \Theta_1) = \sup_{T \in \mathbb{N}}  \sup_{\theta_j \in \Theta_j} \mathbb{E}_{\theta_0, T, \theta_1}[(\tau - T+1)^+ \mid \calF_{T-1}]. 
\end{align}
Here we use the notation $\mathbb{E}_{\theta_0, T, \theta_1}$ to indicate the probability measure over $\calX^\infty$  in which the first $T-1$ observations are from $P_{\theta_0}$ and the ones starting at $T$ are from $P_{\theta_1}$. 

The term $D(\tau,  \Theta_0, \Theta_1)$ is the worst-case (over the choices of the pre- and post-change parameters and the changepoint) detection delay, and $\mathrm{ARL}(\tau, \Theta_0)$  is the worst-case \emph{average run length} when there is no change. 

As in the case of confidence sequences, we can reduce the task of sequential change detection to that of constructing level-$\alpha$ tests for the pre-change class. In particular, consider a testing problem with the null $H_0: \theta \in \Theta_0$, and for any $k \geq 1$, let $N_k$ denote a level-$\alpha$ test for the null $H_0$ constructed using $\{X_i: i \geq k\}$. Then, we can show that 
\begin{align}
    \tau = \inf_{k \geq 1} \{N_k + k -1 \}
\end{align}
is a valid change detection scheme; that is, for this $\tau$, we have $\inf_{\theta_0 \in \Theta_0} \mathbb{E}_{\theta_0}[\tau] \geq 1/\alpha$. The detection delay of the stopping time $\tau$ will depend on the expected stopping time of the individual tests $N_k$, and we will explore this connection in a later lecture. 


\section{Conclusion}
Our discussion reveals that, \emph{in principle}, the problems of sequential estimation via confidence sequences and sequential change detection can be ``reduced to'' constructing power-one tests for appropriate hypotheses~(\emph{in principle}, because the reduction might end up being computationally infeasible). Hence, the key methodological issue for us is to develop general techniques for constructing powerful sequential power-one tests and establish their optimality. That will be the focus of the next few lectures. 


% \newpage
\bibliographystyle{abbrvnat}           % if you need a bibliography
\bibliography{ref}                % assuming yours is named ref.bib


\end{document}